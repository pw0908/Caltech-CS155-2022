\newif\ifshowsolutions
\showsolutionstrue
\documentclass{article}
\usepackage{listings}
\usepackage{amsmath}
%\usepackage{subfigure}
\usepackage{subfig}
\usepackage{amsthm}
\usepackage{amsmath}
\usepackage{amssymb}
\usepackage{graphicx}
\usepackage{mdwlist}
\usepackage[colorlinks=true]{hyperref}
\usepackage{geometry}
\usepackage{titlesec}
\geometry{margin=1in}
\geometry{headheight=2in}
\geometry{top=2in}
\usepackage{palatino}
\usepackage{mathrsfs}
\usepackage{fancyhdr}
\usepackage{paralist}
\usepackage{todonotes}
\setlength{\marginparwidth}{2.15cm}
\usepackage{tikz}
\usetikzlibrary{positioning,shapes,backgrounds}
\usepackage{float} % Place figures where you ACTUALLY want it
\usepackage{comment} % a hack to toggle sections
\usepackage{ifthen}
\usepackage{mdframed}
\usepackage{verbatim}
\usepackage[strings]{underscore}
\usepackage{listings}
\usepackage{bbm}
\rhead{}
\lhead{}

\renewcommand{\baselinestretch}{1.15}

% Shortcuts for commonly used operators
\newcommand{\E}{\mathbb{E}}
\newcommand{\Var}{\operatorname{Var}}
\newcommand{\Cov}{\operatorname{Cov}}
\newcommand{\Bias}{\operatorname{Bias}}
\DeclareMathOperator{\argmin}{arg\,min}
\DeclareMathOperator{\argmax}{arg\,max}

% do not number subsection and below
\setcounter{secnumdepth}{1}

% custom format subsection
\titleformat*{\subsection}{\large\bfseries}

% set up the \question shortcut
\newcounter{question}[section]
\newenvironment{question}[1][]
  {\refstepcounter{question}\par\addvspace{1em}\textbf{Question~\Alph{question}\!
    \ifthenelse{\equal{#1}{}}{}{ [#1 points]}: }}
    {\par\vspace{\baselineskip}}

\newcounter{subquestion}[question]
\newenvironment{subquestion}[1][]
  {\refstepcounter{subquestion}\par\medskip\textbf{\roman{subquestion}.\!
    \ifthenelse{\equal{#1}{}}{}{ [#1 points]:}} }
  {\par\addvspace{\baselineskip}}

\titlespacing\section{0pt}{12pt plus 2pt minus 2pt}{0pt plus 2pt minus 2pt}
\titlespacing\subsection{0pt}{12pt plus 4pt minus 2pt}{0pt plus 2pt minus 2pt}
\titlespacing\subsubsection{0pt}{12pt plus 4pt minus 2pt}{0pt plus 2pt minus 2pt}


\newenvironment{hint}[1][]
  {\begin{em}\textbf{Hint: }}{\end{em}}

\ifshowsolutions
  \newenvironment{solution}[1][]
    {\par\medskip \begin{mdframed}\textbf{Solution~\Alph{question}#1:} \begin{em}}
    {\end{em}\medskip\end{mdframed}\medskip}
  \newenvironment{subsolution}[1][]
    {\par\medskip \begin{mdframed}\textbf{Solution~\Alph{question}#1.\roman{subquestion}:} \begin{em}}
    {\end{em}\medskip\end{mdframed}\medskip}
\else
  \excludecomment{solution}
  \excludecomment{subsolution}
\fi



%%%%%%%%%%%%%%%%%%%%%%%%%%%%%%
% HEADER
%%%%%%%%%%%%%%%%%%%%%%%%%%%%%%

\chead{%
  {\vbox{%
      \vspace{2mm}
      \large
      Machine Learning \& Data Mining \hfill
      Caltech CS/CNS/EE 155 \hfill \\[1pt]
      Homework 6\hfill
      February 2022 \\
    }
  }
}

\begin{document}
\pagestyle{fancy}

%%%%%%%%%%%%%%%%%%%%%%%%%%%%%%
% POLICIES
%%%%%%%%%%%%%%%%%%%%%%%%%%%%%%
\section*{Policies}
\begin{itemize}
  \item \textbf{Due 9 PM, March $2^\text{nd}$}, via Gradescope.
  \item You are free to collaborate on all of the problems, subject to the collaboration policy stated in the syllabus.
  \item In this course, we will be using Google Colab for code submissions. You will need a Google account.
  \item This set uses PyTorch, a Python package for neural networks. We recommend using Google Colab, which comes with PyTorch already installed.
\end{itemize}

\section*{Submission Instructions}
\begin{itemize}
\item Submit your report as a single .pdf file to Gradescope, under "Set 6 Report". 
	\item In the report, \textbf{include any images generated by your code} along with your answers to the questions.
	\item Submit your code by \textbf{sharing a link in your report} to your Google Colab notebook for each problem (see naming instructions below). Make sure to set sharing permissions to at least "Anyone with the link can view". \textbf{Links that can not be run by TAs will not be counted as turned in.} Check your links in an incognito window before submitting to be sure. 
	\item For instructions specifically pertaining to the Gradescope submission process, see \url{https://www.gradescope.com/get_started\#student-submission}.
\end{itemize}

\section*{Google Colab Instructions}
For each notebook, you need to save a copy to your drive.
\begin{enumerate}
	\item Open the github preview of the notebook, and click the icon to open the colab preview.
	\item On the colab preview, go to File $\rightarrow$ Save a copy in Drive.
	\item Edit your file name to “lastname_firstname_set_problem”, e.g.”yue_yisong_set6_prob2.ipynb”
\end{enumerate}

\newpage
%%%%%%%%%%%%%%%%%%%%%%%%%%%%%%
% PROBLEM 1
%%%%%%%%%%%%%%%%%%%%%%%%%%%%%%

\section{Class-Conditional Densities for Binary Data [25 Points, 8 EC Points]}
This problem will test your understanding of probabilistic models, especially Naive Bayes.
Consider a generative classifier for $C$ classes, with class conditional density $p(x | y)$ and a uniform class prior $p(y)$. Suppose all the $D$ features are binary, $x_j \in \{0, 1 \}$. If we assume all of the features are conditionally independent, as in Naive Bayes, we can write:
$$p(x \mid y = c) = \prod_{j=1}^D p(x_j \mid y = c) $$
This requires storing $DC$ parameters. 

Now consider a different model, which we will call the `full' model, in which all the features are fully \textit{dependent}.

\problem[9] Use the chain rule of probability to factorize $p(x \mid y)$, and let $\theta_{xjc} = p(x_j | x_{1, \ldots, j - 1}, y = c)$. Assuming we store each $\theta_{xjc}$, how many parameters are needed to represent this factorization? Use big-O notation. 

\begin{solution}
  Using chain rule, we can write this probability as:
  \begin{equation}
    p(x \mid y = c) = p(x_D\mid x_{D-1},\hdots,x_1,y=c)p(x_{D-1}\mid x_{D-2},\hdots,x_1,y=c)\hdots p(x_1 \mid y=c)
  \end{equation}
  In terms of the variable $\theta_{xjc}$:
  \begin{equation}
    p(x \mid y = c) = \theta_{xDc}\theta_{x(D-1)c}\hdots\theta_{x1c}
  \end{equation}
  or:
  \begin{equation}
    p(x \mid y = c) = \prod_j\theta_{xjc}
  \end{equation}
  Thus, we need $\mathcal{O}(D)$ parameters for this factorisation.

  Given each $\theta_{xjc}$ must store the probability for every prefix of length $j-1$ (i.e. $2^{j-1}$ parameters), the total will be $\mathcal{O}\left(C\sum_{j=0}^{D-1}2^j\right)=\mathcal{O}(C2^D)$. Thus, we need to store $\mathcal{O}(C2^D)$ number of parameters.
\end{solution}

\problem[8] Assume we did no such factorization, and just used the joint probability $p(x \mid y = c)$. How many parameters would we need to estimate in order be able to compute $p(x | y = c)$ for arbitrary $x$ and $c$? How does this compare to your answer from the previous part? Again, use big-O notation.

\begin{solution}
  If we did not do this factorisation, the we would need to store a parameter for each possible classification, $C$, and combination of our binary variables, $2^D$, giving $\mathcal{O}(C2^D)$, the same as the previous case. 
\end{solution}

\problem[4] Assume the number of features $D$ is fixed. Let there be $N$ training cases. If the sample size $N$ is very small, which model (Naive Bayes or full) is likely to give lower test set error, and why?

\begin{solution}
  For a small $N$ training cases, the Naive Bayes model is likely to generalise better as we avoid the risk of overfitting with fewer parameters. This will likely result in a lower testing error. Alternatively, given we have insufficient data, we are better to assume conditional independence between all features.
\end{solution}

\problem[4] If the sample size $N$ is very large, which model (Naive Bayes or full) is likely to give lower test set error, and why?
\begin{solution} 
  Inversly, if we have sufficient data, then the risk of overfitting is reduced and we can try to account for the inter-dependence of features for a given input. Trying to limit the number of parameters in the Naive Bayes approach will like result in underfitting, meaning we would expect a lower error in the testing set for the full model.
\end{solution}

\problem[8 EC] Assume all the parameter estimates have been computed. What is the computational complexity of making a prediction, i.e. computing $p(y \mid x)$, using Naive Bayes for a single test case? What is the computation complexity of making a prediction with the full model? In justifying your answer for the full model, choose either the implementation in 1A or 1B and state your choice. For the full-model case, assume that converting a $D$-bit vector to an array index is an $O(D)$ operation.  Also, recall that we have assumed a uniform class prior. 

\begin{solution}
  We know that:
  \begin{equation}
    p(y | x)= p(x | y) \frac{p(y)}{p(x)}
  \end{equation}
  Where, for the Naive Bayes model, computing $p(x)$ is an $\mathcal{O}(D)$ operation, computing $p(y)$ is an $\mathcal{O}(1)$ operation and computing $p(x | y)=\sum_cp(x | y=c)p(y=c)=\sum_c\prod_dp(x_d | y=c)p(y=c)$ is an $\mathcal{O}(CD)$ operation. Thus, the overall computational time is $\mathcal{O}(CD)$.

  For the full model, all the above remains true except for computing $p(x | y)$ where we convert the $D$-bit vector to an array index ($\mathcal{O}(D)$); after this, it is just a matter of obtaining $p(x | y=c)$ for a corresponding value of $y$. As such, the overall computational time is $\mathcal{O}(D+C)$.
\end{solution}


\newpage
\section{Sequence Prediction [75 Points]}

In this problem, we will explore some of the various algorithms associated with Hidden Markov Models (HMMs), as discussed in lecture.
We have also uploaded a note on HMMs to the github that might be helpful.

\subsection{Sequence Prediction}

These next few problems will require extensive coding, so be sure to start early! 
\begin{itemize}
  \item You will write an implementation for the hidden Markov model in the cell for \texttt{HMM Code} in the notebook given to you, within the appropriate functions where indicated. There should be no need to write additional functions or use NumPy in your implementation, but feel free to do so if you would like.
  \item You can (and should!) use the helper cells for each of the subproblems in the notebook, namely \texttt{2A}, \texttt{2Bi}, \texttt{2Bii}, \texttt{2C}, \texttt{2D}, and \texttt{2F}. These can be used to run and check your implementations for each of the corresponding problems. The cells provide useful output in an easy-to-read format. There is no need to modify these cells.
  \item Lastly, the cell for \texttt{Utility} contains some functions used for loading data directly from the class github repository. There is no need to modify this cell. \\
\end{itemize}

The supplementary data folder of the class github repository contains 6 files titled \texttt{sequence_data0.txt}, \texttt{sequence_data1.txt}, ... , \texttt{sequence_data5.txt}. Each file specifies a \textbf{trained} HMM. The first row contains two tab-delimited numbers: the number of states $Y$ and the number of types of observations $X$ (i.e. the observations are $0, 1, . . . , X - 1$). The next $Y$ rows of $Y$ tab-delimited floating-point numbers describe the state transition matrix. Each row represents the current state, each column represents a state to transition to, and each entry represents the probability of that transition occurring. The next $Y$ rows of $X$ tab-delimited floating-point numbers describe the output emission matrix, encoded analogously to the state transition matrix. The file ends with 5 possible emissions from that HMM. \\

The supplementary data folder also contains one additional file titled \texttt{ron.txt}. This is used in problems 2C and 2D and is explained in greater detail there. 
\indent\problem[10] % indent for consistency
For each of the six trained HMMs, find the max-probability state sequence for each of the five input sequences at the end of the corresponding file. To complete this problem, you will have to implement the Viterbi algorithm (in \texttt{viterbi()} of the \texttt{HiddenMarkovModel} object). Write your implementation well, as we will be reusing it in a later problem. See the end of problem 2B for a big hint! Note that you do not need to worry about underflow in this part.

In your report, show your results on the 6 files. (Copy-pasting the results of the cell for \texttt{2A} suffices.)
\begin{solution}
  Link: \url{https://colab.research.google.com/drive/1pL8tB6NVid0nKR_rG-9xLFb4Hy56jIWL?usp=sharing}
File \#0:\\
Emission Sequence             Max Probability State Sequence\\
\footnotesize\#\#\#\#\#\#\#\#\#\#\#\#\#\#\#\#\#\#\#\#\#\#\#\#\#\#\#\#\#\#\#\#\#\#\#\#\#\#\#\#\#\#\#\#\#\#\#\#\#\#\#\#\#\#\#\#\#\#\#\#\\
25421                         31033          \\               
01232367534                   22222100310    \\               
5452674261527433              1031003103222222       \\       
7226213164512267255           1310331000033100310      \\     
0247120602352051010255241     2222222222222222222222103    \\ 

File \#1:\\
Emission Sequence             Max Probability State Sequence\\
\footnotesize\#\#\#\#\#\#\#\#\#\#\#\#\#\#\#\#\#\#\#\#\#\#\#\#\#\#\#\#\#\#\#\#\#\#\#\#\#\#\#\#\#\#\#\#\#\#\#\#\#\#\#\#\#\#\#\#\#\#\#\#\\
77550                         22222      \\                   
7224523677                    2222221000 \\                   
505767442426747               222100003310031        \\       
72134131645536112267          10310310000310333100     \\     
4733667771450051060253041     2221000003222223103222223  \\   

File \#2:\\
Emission Sequence             Max Probability State Sequence\\
\footnotesize\#\#\#\#\#\#\#\#\#\#\#\#\#\#\#\#\#\#\#\#\#\#\#\#\#\#\#\#\#\#\#\#\#\#\#\#\#\#\#\#\#\#\#\#\#\#\#\#\#\#\#\#\#\#\#\#\#\#\#\#\\
60622                         11111     \\                    
4687981156                    2100202111          \\          
815833657775062               021011111111111          \\     
21310222515963505015          02020111111111111021        \\  
6503199452571274006320025     1110202111111102021110211     \\

File \#3:\\
Emission Sequence             Max Probability State Sequence\\
\footnotesize\#\#\#\#\#\#\#\#\#\#\#\#\#\#\#\#\#\#\#\#\#\#\#\#\#\#\#\#\#\#\#\#\#\#\#\#\#\#\#\#\#\#\#\#\#\#\#\#\#\#\#\#\#\#\#\#\#\#\#\#\\
13661                         00021      \\                   
2102213421                    3131310213     \\               
166066262165133               133333133133100    \\           
53164662112162634156          20000021313131002133   \\       
1523541005123230226306256     1310021333133133133133133  \\   

File \#4:
Emission Sequence             Max Probability State Sequence\\
\footnotesize\#\#\#\#\#\#\#\#\#\#\#\#\#\#\#\#\#\#\#\#\#\#\#\#\#\#\#\#\#\#\#\#\#\#\#\#\#\#\#\#\#\#\#\#\#\#\#\#\#\#\#\#\#\#\#\#\#\#\#\#\\
23664                         01124           \\              
3630535602                    0111201112       \\             
350201162150142               011244012441112      \\         
00214005402015146362          11201112412444011112   \\       
2111266524665143562534450     2012012424124011112411124   \\  

File \#5:\\
Emission Sequence             Max Probability State Sequence\\
\footnotesize\#\#\#\#\#\#\#\#\#\#\#\#\#\#\#\#\#\#\#\#\#\#\#\#\#\#\#\#\#\#\#\#\#\#\#\#\#\#\#\#\#\#\#\#\#\#\#\#\#\#\#\#\#\#\#\#\#\#\#\#\\
68535                         10111       \\                  
4546566636                    1111111111       \\             
638436858181213               110111010000011      \\         
13240338308444514688          00010000000111111100    \\      
0111664434441382533632626     2111111111111100111110101   \\  

\end{solution}
\indent\problem[17] % indent for consistency
For each of the six trained HMMs, find the probabilities of emitting the five input sequences at the end of the corresponding file. To complete this problem, you will have to implement the Forward algorithm and the Backward algorithm. You may assume that the initial state is randomly selected along a uniform distribution (the starting state transition probabilities are defined in \texttt{self.A\_start} in \texttt{HiddenMarkovModel}). Again, write your implementation well, as we will be reusing it in a later problem. \\

Note that the probability of emitting an input sequence can be found by using either the $\alpha$ vectors from the Forward algorithm or the $\beta$ vectors from the Backward algorithm. You don't need to worry about this, as it is done for you in \texttt{probability\_alphas()} and \texttt{probability\_betas()}.

Implement the Forward algorithm. In your report, show your results on the 6 files. \\
Implement the Backward algorithm. In your report, show your results on the 6 files. \\

After you complete problems 2A and 2B, you can compare your results (the probabilities from the forward and backward algorithms should be the same) for the file titled \texttt{sequence_data0.txt} with the values given in the table below:
\begin{center}
  \begin{tabular}{ l | l |l | l }
Dataset & Emission Sequence & Max-probability State Sequence & Probability of Sequence  \\ \hline
0 & 25421                      &  31033           & 4.537e-05\\
0 & 01232367534                &  22222100310       & 1.620e-11\\
0 & 5452674261527433           &  1031003103222222      & 4.348e-15\\
0 & 7226213164512267255        &  1310331000033100310   & 4.739e-18\\
0 & 0247120602352051010255241  &  2222222222222222222222103 & 9.365e-24
 \\ \hline
 \end{tabular}
\end{center}

\begin{solution}
  \textbf{Forward algorithm:}\\
  File \#0:\\
  Emission Sequence             Probability of Emitting Sequence\\
  \footnotesize \#\#\#\#\#\#\#\#\#\#\#\#\#\#\#\#\#\#\#\#\#\#\#\#\#\#\#\#\#\#\#\#\#\#\#\#\#\#\#\#\#\#\#\#\#\#\#\#\#\#\#\#\#\#\#\#\#\#\#\#\#\#\#\#\#\\
  25421                         4.537e-05 \\
  01232367534                   1.620e-11 \\
  5452674261527433              4.348e-15 \\
  7226213164512267255           4.739e-18 \\
  0247120602352051010255241     9.365e-24 \\
  
  File \#1:\\
  Emission Sequence             Probability of Emitting Sequence\\
  \footnotesize \#\#\#\#\#\#\#\#\#\#\#\#\#\#\#\#\#\#\#\#\#\#\#\#\#\#\#\#\#\#\#\#\#\#\#\#\#\#\#\#\#\#\#\#\#\#\#\#\#\#\#\#\#\#\#\#\#\#\#\#\#\#\#\#\#\\
  77550                         1.181e-04 \\
  7224523677                    2.033e-09 \\
  505767442426747               2.477e-13 \\
  72134131645536112267          8.871e-20 \\
  4733667771450051060253041     3.740e-24 \\
  
  File \#2:\\
  Emission Sequence             Probability of Emitting Sequence\\
  \footnotesize \#\#\#\#\#\#\#\#\#\#\#\#\#\#\#\#\#\#\#\#\#\#\#\#\#\#\#\#\#\#\#\#\#\#\#\#\#\#\#\#\#\#\#\#\#\#\#\#\#\#\#\#\#\#\#\#\#\#\#\#\#\#\#\#\#\\
  60622                         2.088e-05 \\
  4687981156                    5.181e-11 \\
  815833657775062               3.315e-15 \\
  21310222515963505015          5.126e-20 \\
  6503199452571274006320025     1.297e-25 \\
  
  File \#3:\\
  Emission Sequence             Probability of Emitting Sequence\\
  \footnotesize \#\#\#\#\#\#\#\#\#\#\#\#\#\#\#\#\#\#\#\#\#\#\#\#\#\#\#\#\#\#\#\#\#\#\#\#\#\#\#\#\#\#\#\#\#\#\#\#\#\#\#\#\#\#\#\#\#\#\#\#\#\#\#\#\#\\
  13661                         1.732e-04 \\
  2102213421                    8.285e-09 \\
  166066262165133               1.642e-12 \\
  53164662112162634156          1.063e-16 \\
  1523541005123230226306256     4.535e-22 \\
  
  File \#4:\\
  Emission Sequence             Probability of Emitting Sequence\\
  \footnotesize \#\#\#\#\#\#\#\#\#\#\#\#\#\#\#\#\#\#\#\#\#\#\#\#\#\#\#\#\#\#\#\#\#\#\#\#\#\#\#\#\#\#\#\#\#\#\#\#\#\#\#\#\#\#\#\#\#\#\#\#\#\#\#\#\#\\
  23664                         1.141e-04 \\
  3630535602                    4.326e-09 \\
  350201162150142               9.793e-14 \\
  00214005402015146362          4.740e-18 \\
  2111266524665143562534450     5.618e-22 \\
  
  File \#5:\\
  Emission Sequence             Probability of Emitting Sequence\\
  \footnotesize \#\#\#\#\#\#\#\#\#\#\#\#\#\#\#\#\#\#\#\#\#\#\#\#\#\#\#\#\#\#\#\#\#\#\#\#\#\#\#\#\#\#\#\#\#\#\#\#\#\#\#\#\#\#\#\#\#\#\#\#\#\#\#\#\#\\
  68535                         1.322e-05 \\
  4546566636                    2.867e-09 \\
  638436858181213               4.323e-14 \\
  13240338308444514688          4.629e-18 \\
  0111664434441382533632626     1.440e-22 \\
  
  \textbf{Backward algorithm:}\\
  File \#0:\\
  Emission Sequence             Probability of Emitting Sequence\\
  \footnotesize \#\#\#\#\#\#\#\#\#\#\#\#\#\#\#\#\#\#\#\#\#\#\#\#\#\#\#\#\#\#\#\#\#\#\#\#\#\#\#\#\#\#\#\#\#\#\#\#\#\#\#\#\#\#\#\#\#\#\#\#\#\#\#\#\#\\
  25421                         4.537e-05 \\
  01232367534                   1.620e-11 \\
  5452674261527433              4.348e-15 \\
  7226213164512267255           4.739e-18 \\
  0247120602352051010255241     9.365e-24 \\
  
  File \#1:\\
  Emission Sequence             Probability of Emitting Sequence\\
  \footnotesize \#\#\#\#\#\#\#\#\#\#\#\#\#\#\#\#\#\#\#\#\#\#\#\#\#\#\#\#\#\#\#\#\#\#\#\#\#\#\#\#\#\#\#\#\#\#\#\#\#\#\#\#\#\#\#\#\#\#\#\#\#\#\#\#\#\\
  77550                         1.181e-04 \\
  7224523677                    2.033e-09 \\
  505767442426747               2.477e-13 \\
  72134131645536112267          8.871e-20 \\
  4733667771450051060253041     3.740e-24 \\
  
  File \#2:\\
  Emission Sequence             Probability of Emitting Sequence\\
  \footnotesize \#\#\#\#\#\#\#\#\#\#\#\#\#\#\#\#\#\#\#\#\#\#\#\#\#\#\#\#\#\#\#\#\#\#\#\#\#\#\#\#\#\#\#\#\#\#\#\#\#\#\#\#\#\#\#\#\#\#\#\#\#\#\#\#\#\\
  60622                         2.088e-05 \\
  4687981156                    5.181e-11 \\
  815833657775062               3.315e-15 \\
  21310222515963505015          5.126e-20 \\
  6503199452571274006320025     1.297e-25 \\
  
  File \#3:\\
  Emission Sequence             Probability of Emitting Sequence\\
  \footnotesize \#\#\#\#\#\#\#\#\#\#\#\#\#\#\#\#\#\#\#\#\#\#\#\#\#\#\#\#\#\#\#\#\#\#\#\#\#\#\#\#\#\#\#\#\#\#\#\#\#\#\#\#\#\#\#\#\#\#\#\#\#\#\#\#\#\\
  13661                         1.732e-04 \\
  2102213421                    8.285e-09 \\
  166066262165133               1.642e-12 \\
  53164662112162634156          1.063e-16 \\
  1523541005123230226306256     4.535e-22 \\
  
  File \#4:\\
  Emission Sequence             Probability of Emitting Sequence\\
  \footnotesize \#\#\#\#\#\#\#\#\#\#\#\#\#\#\#\#\#\#\#\#\#\#\#\#\#\#\#\#\#\#\#\#\#\#\#\#\#\#\#\#\#\#\#\#\#\#\#\#\#\#\#\#\#\#\#\#\#\#\#\#\#\#\#\#\#\\
  23664                         1.141e-04 \\
  3630535602                    4.326e-09 \\
  350201162150142               9.793e-14 \\
  00214005402015146362          4.740e-18 \\
  2111266524665143562534450     5.618e-22 \\
  
  File \#5:\\
  Emission Sequence             Probability of Emitting Sequence\\
  \footnotesize \#\#\#\#\#\#\#\#\#\#\#\#\#\#\#\#\#\#\#\#\#\#\#\#\#\#\#\#\#\#\#\#\#\#\#\#\#\#\#\#\#\#\#\#\#\#\#\#\#\#\#\#\#\#\#\#\#\#\#\#\#\#\#\#\#\\
  68535                         1.322e-05 \\
  4546566636                    2.867e-09 \\
  638436858181213               4.323e-14 \\
  13240338308444514688          4.629e-18 \\
  0111664434441382533632626     1.440e-22 \\
  
  
\end{solution}

\subsection{HMM Training}
Ron is an avid music listener, and his genre preferences at any given time depend on his mood. Ron's possible moods are happy, mellow, sad, and angry. Ron experiences one mood per day (as humans are known to do) and chooses one of ten genres of music to listen to that day depending on his mood. \\

Ron's roommate, who is known to take to odd hobbies, is interested in how Ron's mood affects his music selection, and thus collects data on Ron's mood and music selection for six years (2190 data points). This data is contained in the supplementary file \texttt{ron.txt}. Each row contains two tab-delimited strings: Ron's mood and Ron's genre preference that day. The data is split into 12 sequences, each corresponding to half a year's worth of observations. The sequences are separated by a row containing only the character \texttt{-}.
\noindent\problem[10] % indent for consistency
Use a single M-step to train a supervised Hidden Markov Model on the data in \texttt{ron.txt}. What are the learned state transition and output emission matrices?

Tip: the \((1,1)\) entry of your transition matrix should be \verb|2.833e-01|, and the \((1,1)\) entry of your observation matrix should be \verb|1.486e-01|.
\begin{solution}
  Transition Matrix:\\
{\footnotesize \#\#\#\#\#\#\#\#\#\#\#\#\#\#\#\#\#\#\#\#\#\#\#\#\#\#\#\#\#\#\#\#\#\#\#\#\#\#\#\#\#\#\#\#\#\#\#\#\#\#\#\#\#\#\#\#\#\#\#\#\#\#\#\#\#\\}
2.833e-01   4.714e-01   1.310e-01   1.143e-01   \\
2.321e-01   3.810e-01   2.940e-01   9.284e-02   \\
1.040e-01   9.760e-02   3.696e-01   4.288e-01   \\
1.883e-01   9.903e-02   3.052e-01   4.075e-01   \\


Observation Matrix:  \\
{\footnotesize \#\#\#\#\#\#\#\#\#\#\#\#\#\#\#\#\#\#\#\#\#\#\#\#\#\#\#\#\#\#\#\#\#\#\#\#\#\#\#\#\#\#\#\#\#\#\#\#\#\#\#\#\#\#\#\#\#\#\#\#\#\#\#\#\#\\}
1.486e-01   2.288e-01   1.533e-01   1.179e-01   4.717e-02   5.189e-02   2.830e-02   1.297e-01   9.198e-02   2.358e-03   \\
1.062e-01   9.653e-03   1.931e-02   3.089e-02   1.699e-01   4.633e-02   1.409e-01   2.394e-01   1.371e-01   1.004e-01   \\
1.194e-01   4.299e-02   6.529e-02   9.076e-02   1.768e-01   2.022e-01   4.618e-02   5.096e-02   7.803e-02   1.274e-01   \\
1.694e-01   3.871e-02   1.468e-01   1.823e-01   4.839e-02   6.290e-02   9.032e-02   2.581e-02   2.161e-01   1.935e-02 \\
\end{solution}
\indent\problem[15] % indent for consistency
Now suppose that Ron has a third roommate who is also interested in how Ron's mood affects his music selection. This roommate is lazier than the other one, so he simply steals the first roommate's data. Unfortunately, he only manages to grab half the data, namely, Ron's choice of music for each of the 2190 days.

In this problem, we will train an unsupervised Hidden Markov Model on this data. Recall that unsupervised HMM training is done using the Baum-Welch algorithm and will require repeated EM steps. For this problem, we will use 4 hidden states and run the algorithm for 1000 iterations. The transition and observation matrices are initialized for you in the helper functions \texttt{supervised\_learning()} and \texttt{unsupervised\_learning()} such that they are random and normalized.

What are the learned state transition and output emission matrices? Please report the result using random seed state 1, as is done by default in the notebook.

Tips for debugging:
\begin{itemize}
  \item The rows of the state transition and output emitting matrices should sum to 1.
  \item Your matrices should not change drastically every iteration.
  \item After many iterations, your matrices should converge.
  \item If you used random seed 1 for this computation (as is done by default in the notebook), the \((1,1)\) entry of the state transition matrix should be \verb|5.075e-01|, and the \((1,1)\) entry of the output emission matrix should be \verb|1.117e-01|.
\end{itemize}

\begin{solution}
  Transition Matrix:\\
{\footnotesize \#\#\#\#\#\#\#\#\#\#\#\#\#\#\#\#\#\#\#\#\#\#\#\#\#\#\#\#\#\#\#\#\#\#\#\#\#\#\#\#\#\#\#\#\#\#\#\#\#\#\#\#\#\#\#\#\#\#\#\#\#\#\#\#\#}\\
5.075e-01   4.596e-01   6.533e-09   3.292e-02   \\
3.127e-03   2.107e-04   9.964e-01   2.733e-04   \\
1.195e-09   6.886e-02   9.686e-16   9.311e-01   \\
6.203e-01   3.796e-01   1.555e-05   1.579e-04   \\


Observation Matrix:  \\
{\footnotesize \#\#\#\#\#\#\#\#\#\#\#\#\#\#\#\#\#\#\#\#\#\#\#\#\#\#\#\#\#\#\#\#\#\#\#\#\#\#\#\#\#\#\#\#\#\#\#\#\#\#\#\#\#\#\#\#\#\#\#\#\#\#\#\#\#}\\
1.117e-01   1.525e-01   7.740e-02   1.975e-02   1.594e-01   4.574e-13   3.556e-16   2.475e-01   1.139e-01   1.180e-01   \\
1.205e-01   2.548e-15   1.103e-01   1.751e-01   3.656e-04   2.190e-01   1.002e-01   6.178e-02   1.323e-01   8.053e-02   \\
1.276e-01   2.665e-02   5.788e-02   1.682e-01   1.700e-01   6.969e-02   1.254e-01   3.940e-02   1.627e-01   5.244e-02   \\
1.918e-01   8.206e-02   1.376e-01   8.725e-02   1.152e-01   1.209e-01   1.033e-01   3.101e-02   1.308e-01   5.847e-38   \\
\end{solution}

\problem[5] How do the transition and emission matrices from 2C and 2D compare? Which do you think provides a more accurate representation of Ron's moods and how they affect his music choices? Justify your answer. Suggest one way that we may be able to improve the method (supervised or unsupervised) that you believe produces the less accurate representation. 

\begin{solution}
  The most obvious distinction between the matrices obtained for supervised and unsurpervised HMM is that the latter is more-sparse, with values ranging from $10^{-16}$ to $10^{-1}$ whilst, in the former, most values are between $10^{-2}$ and $10^{-1}$ in the transition matrix. We do note that the supervised model always produces the same transition and observation matrices for a given training set as the minimum is global (obtained from a closed-form solution). On the other hand, by using the Baum-Welch algorithm, unsupervised learning can only find local minima which vary based on our initial starting position. If we assumed the training set adequately represents Ron's mood and the impact on his music choice, then the supervised model is more likely to give an accurate representation since it obtains the global optimum. We have no guarantee that the local minima obtained in the unsupervised model correctly represent his true behaviour (due to hidden states). We could improve our likelihood of ending up in the global minima in our unsupervised model by using more-informed guesses which are closer to the global minima. 
\end{solution}

\subsection{Sequence Generation}
Hidden Markov Models fall under the umbrella of generative models and therefore can be used to not only predict sequential data, but also to generate it.

\problem[5] Run the cell for this problem. The code in the cell loads the trained HMMs from the files titled \texttt{sequence_data0.txt}, $\ldots$ , \texttt{sequence_data5.txt} and uses the six models to probabilistically generate five sequences of emissions from each model, each of length 20. In your report, show your results. 

\begin{solution}
File \#0:\\
Generated Emission            \\
{\footnotesize \#\#\#\#\#\#\#\#\#\#\#\#\#\#\#\#\#\#\#\#\#\#\#\#\#\#\#\#\#\#\#\#\#\#\#\#\#\#\#\#\#\#\#\#\#\#\#\#\#\#\#\#\#\#\#\#\#\#\#\#\#\#\#\#\#}\\
61372121576400542636    \\      
65257035512152557265    \\      
22503513757650076565    \\      
24302052542075224453    \\      
74164372540666312532    \\      

File \#1:\\
Generated Emission  \\          
{\footnotesize \#\#\#\#\#\#\#\#\#\#\#\#\#\#\#\#\#\#\#\#\#\#\#\#\#\#\#\#\#\#\#\#\#\#\#\#\#\#\#\#\#\#\#\#\#\#\#\#\#\#\#\#\#\#\#\#\#\#\#\#\#\#\#\#\#}\\
67224254254640221044    \\      
53145572605255061365   \\       
03120652632260563554   \\       
57754107154004157145   \\       
50443620741222175553   \\       

File \#2:\\
Generated Emission      \\      
{\footnotesize \#\#\#\#\#\#\#\#\#\#\#\#\#\#\#\#\#\#\#\#\#\#\#\#\#\#\#\#\#\#\#\#\#\#\#\#\#\#\#\#\#\#\#\#\#\#\#\#\#\#\#\#\#\#\#\#\#\#\#\#\#\#\#\#\#}\\
23492277012312492009     \\     
92715253263463500777     \\     
00206169901079655318     \\     
34658526761560627594     \\     
98316963019608446826     \\     

File \#3:\\
Generated Emission      \\      
{\footnotesize \#\#\#\#\#\#\#\#\#\#\#\#\#\#\#\#\#\#\#\#\#\#\#\#\#\#\#\#\#\#\#\#\#\#\#\#\#\#\#\#\#\#\#\#\#\#\#\#\#\#\#\#\#\#\#\#\#\#\#\#\#\#\#\#\#}\\
53324006315616155155    \\      
61131140351050201522    \\      
26364625661106214604    \\      
02325053164160635305    \\      
20066023620261022165    \\      

File \#4:\\
Generated Emission   \\         
{\footnotesize \#\#\#\#\#\#\#\#\#\#\#\#\#\#\#\#\#\#\#\#\#\#\#\#\#\#\#\#\#\#\#\#\#\#\#\#\#\#\#\#\#\#\#\#\#\#\#\#\#\#\#\#\#\#\#\#\#\#\#\#\#\#\#\#\#}\\
63333124051064432222          \\
54046611655614306333          \\
30164016126063513444          \\
66156363216364120060          \\
44664503206213132626          \\

File \#5:\\
Generated Emission            \\
{\footnotesize \#\#\#\#\#\#\#\#\#\#\#\#\#\#\#\#\#\#\#\#\#\#\#\#\#\#\#\#\#\#\#\#\#\#\#\#\#\#\#\#\#\#\#\#\#\#\#\#\#\#\#\#\#\#\#\#\#\#\#\#\#\#\#\#\#}\\
23118681064868303466          \\
68845455633644828521          \\
30418103230140186766          \\
10567813866403462083          \\
82316383453840668047          \\
\end{solution}

\subsection{Visualization \& Analysis}

Once you have implemented the HMM code part of the notebook, load and run the cells for the following subproblems. Here you will apply the HMM you have implemented to the Constitution. There is no coding required for this part, only analysis.

Answer the following problems in the context of the visualizations in the notebook.

\indent\problem[3] % indent for consistency
What can you say about the sparsity of the trained $A$ and $O$ matrices? How does this sparsity affect the transition and observation behaviour at each state?
\begin{solution}
  Although both are sparse, $O$ is generally more-sparse than $A$. A zero element in row $i$ of matrix $A$ indicates that the state $i$ has a zero probably of transitioning to the state corresponding to that column (thus, a sparse row $i$ means that there are fewer states to transition to). Similarly, a zero element in row $i$ of matrix $O$ indicates a zero probability of generating the observation corresponding to that column from state $i$; thus a sparse row $i$ indicates that few observations can be generated from state $i$. If all elements in a column of $O$ are very small valued, it indicates that this observation is very unlikely to be generated. As $O$ is more-sparse than $A$, it indicates that the relationship between states and observations is stronger than it is between states (transitioning). 
\end{solution}

\indent\problem[5] % indent for consistency
How do the sample emission sentences from the HMM change as the number of hidden states is increased? What happens in the special case where there is only one hidden state? In general, when the number of hidden states is unknown while training an HMM for a fixed observation set, can we increase the training data likelihood by allowing more hidden states?

\begin{solution}
  As we increase the number of hidden states, the sample emission begin to resemble something akin to an actual sentence from the constitution. In the case when there is only one hidden layer, the sample emission sentence doesn't make any sense; this is because the words are picked completely at random given, with one hidden layer, all words have the same probability of being picked (weighted by the frequency at which they occur in the original text). This partly explains why the word `president' appears twice. 

  We can indeed increase training data likelihood by increasing the number of allowed hidden states. Doing so results in sparser transition and observation matrices meaning there is a stronger relationship between states and observations, and other states. In the extreme case where we have the same number of states as observations, our observation matrix should approach some sort of identity matrix (with some columns permuted), meaning each state is perfectly associated with one observation, maximising our training likelihood.
\end{solution}


\indent\problem[5] % indent for consistency
Pick a state that you find semantically meaningful, and analyze this state and its wordcloud. What does this state represent? How does this state differ from the other states? Back up your claim with a few key words from the wordcloud.
\begin{solution}
  State 3 appears to be the most semantically meaningful, where most of the words in this state are either numbers (two, three, four, seven, eight, thirty, etc.) and words that usually involve counting (days, ballot, grant, pay, etc.). Most of the other states appear to be semi-random, liked to the frequency of the word or words just linked to the consistution. Most of the words in state 3 serve very similar grammatical purposes.
\end{solution}



\end{document}
